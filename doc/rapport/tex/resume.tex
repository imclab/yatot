\section*{Résumé}
\label{sec:resume}
\addcontentsline{toc}{section}{Résumé}

% De toute votre thèse, cette partie sera celle qui aura le publique le plus grand.
% Il est mieux de l'écrire vers la fin, mais pas à la dernière minute parce que vous aurez besoin d'en faire plusieurs jets.
% En plus, le résumé a une telle importance que ça vaut vraiment la peine de demander à un collègue ou copain anglophone d'éditer votre version.
% Le résumé devrait être une distillation de la thèse: une description concise du problème(s) adressé(s), votre méthode pour le(s) résoudre, vos résultats et conclusions.
% Un résumé doit être indépendant.
% D'habitude il ne contient pas de références.
% Quand une référence est nécessaire, ses détails devraient être inclus dans le texte du résumé.
% Vérifiez la limite de longueur avec votre fac. 

\begin{todo}

\end{todo}

\section*{Mots-clés}
\label{sec:motsCles}

Traitement Automatique du Langage Naturel, réseaux lexico-sémantique, analyse
sémantique.